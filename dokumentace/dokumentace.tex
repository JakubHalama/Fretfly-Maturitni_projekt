% ŠABLONA PRO PSANÍ ZÁVĚREČNÉ STUDIJNÍ PRÁCE
%%%%%%%%%%%%%%%%%%%%%%%%%%%%%%%%%%%%%%%%%%%%
% Autor: Jakub Dokulil (kubadokulil99@gmail.com)
% Tato šablona byla vytvořena tak, aby pomocí ní mohli v systému LaTeX soutěžící sázet své práce a zároveň odpovídala požadavkům na formátování vyplývajícím z wordové šablony umístěné na webu soc.cz.
%
\documentclass[12pt, a4paper,hidelinks
%%oneside,      %% -- odkomentujte, pokud chcete svou práci mít pouze jednostrannou, mezera pro hřbet pak automaticky bude pouze na levé straně
%%twoside,        %% -- pro oboustranné práce, mezera pro hřbet následně střídá strany.
openright
]{report}

%% Nutné balíčky a nastavení
%%%%%%%%%%%%%%%%%%%%%%%%%%%%

%% Proměnné
\newcommand\obor{INFORMAČNÍ TECHNOLOGIE} %% -- napiš číslo a název tvého oboru
\newcommand\kodOboru{18-20-M/01} %% -- napiš číslo a název tvého oboru
\newcommand\zamereni{se zaměřením na počítačové sítě a programování} %% -- napiš číslo a název tvého oboru
\newcommand\skola{Střední škola průmyslová a umělecká, Opava} %% vyplň název školy
\newcommand\trida{IT4} %% vyplň jméno svého konzultanta
\newcommand\jmenoAutora{Jakub Halama}  %% vyplň své jméno
\newcommand\skolniRok{2025/26} %% vyplň rok
\newcommand\datumOdevzdani{5. 1. 2026} %% vyplň rok
\newcommand\nazevPrace{FretFly: Kytarová aplikace ve frameworku Flutter} %% vyplň název své práce


\title{\nazevPrace} %% -- Název tvé práce
\author{\jmenoAutora} %% -- tvé jméno
\date{\datumOdevzdani} %% -- rok

\usepackage[top=2.5cm, bottom=2.5cm, left=3.5cm, right=1.5cm]{geometry} %% nastaví okraje, left -- vnitřní okraj, right -- vnější okraj

\usepackage[czech]{babel} %% balík babel pro sazbu v češtině
\usepackage[utf8]{inputenc} %% balíky pro kódování textu
\usepackage[T1]{fontenc}
\usepackage{cmap} %% balíček zajišťující, že vytvořené PDF bude prohledávatelné a kopírovatelné

\usepackage{graphicx} %% balík pro vkládání obrázků

\usepackage{subcaption} %% balíček pro vkládání podobrázků
\usepackage{float}
\usepackage[hidelinks]{hyperref} %% balíček, který v PDF vytváří odkazy

\linespread{1.25} %% řádkování
\setlength{\parskip}{0.5em} %% odsazení mezi odstavci


\usepackage[pagestyles]{titlesec} %% balíček pro úpravu stylu kapitol a sekcí
\titleformat{\chapter}[block]{\scshape\bfseries\LARGE}{\thechapter}{10pt}{\vspace{0pt}}[\vspace{-22pt}]
\titleformat{\section}[block]{\scshape\bfseries\Large}{\thesection}{10pt}{\vspace{0pt}}
\titleformat{\subsection}[block]{\bfseries\large}{\thesubsection}{10pt}{\vspace{0pt}}

\usepackage{tocloft} % Balíček umožní přizpůsobit vzhled tabulky obsahu
\setlength{\cftbeforechapskip}{0pt}  % Menší rozestup pro kapitoly
\setlength{\cftbeforesecskip}{0pt}   % Menší rozestup pro sekce

\setcounter{secnumdepth}{2}
\setcounter{tocdepth}{1}
\usepackage{fancyhdr}
\pagestyle{fancy}
\fancyhf{}
\renewcommand{\headrulewidth}{0pt}
\fancyfoot[C]{\thepage}



\usepackage{booktabs}
\usepackage{microtype}
\usepackage{url}

%% Balíčky co se můžou hodit :) 
%%%%%%%%%%%%%%%%%%%%%%%%%%%%%%%

\usepackage{pdfpages} %% Balíček umožňující vkládat stránky z PDF souborů, 

\usepackage{upgreek} %% Balíček pro sazbu stojatých řeckých písmen, třeba u jednotky mikrometr. Například stojaté mí: \upmu, stojaté pí: \uppi

\usepackage{amsmath}    %% Balíčky amsmath a amsfonts 
\usepackage{amsfonts}   %% pro sazbu matematických symbolů
\usepackage{esint}     %% pro sazbu různých integrálů (např \oiint)
\usepackage{mathrsfs}
\usepackage{helvet} % Helvet font
\usepackage{mathptmx} % Times New Roman
\usepackage{Oswald} % Oswald font


%% makra pro sazbu matematiky
\newcommand{\dif}{\mathrm{d}} %% makro pro sazbu diferenciálu, místo toho
%% abych musel psát '\mathrm{d}' mi stačí napsat '\dif' což je mnohem 
%% kratší a mohu si tak usnadnit práci

\let\oldchapter\chapter
\renewcommand{\chapter}{
	\clearpage
	\pagestyle{fancy}
	\oldchapter
}

\usepackage{listings}
\usepackage{xcolor}
\usepackage[czech]{babel}
\usepackage[T1]{fontenc}
\usepackage[utf8]{inputenc} 

\renewcommand{\lstlistingname}{Kód}% Listing -> Algorithm
\renewcommand{\lstlistlistingname}{Seznam programových kódů}% List of Listings -> List of Algorithms


%copied code start
\definecolor{mediumgray}{rgb}{0.3, 0.4, 0.4}
\definecolor{mediumblue}{rgb}{0.0, 0.0, 0.8}
\definecolor{forestgreen}{rgb}{0.13, 0.55, 0.13}
\definecolor{darkviolet}{rgb}{0.58, 0.0, 0.83}
\definecolor{royalblue}{rgb}{0.25, 0.41, 0.88}
\definecolor{crimson}{rgb}{0.86, 0.8, 0.24}

\lstdefinestyle{csh}{
	language=C,
	backgroundcolor=\color{white},
	basicstyle=\ttfamily,
	breakatwhitespace=false,
	breaklines=false,
	captionpos=b,
	columns=fullflexible,
	commentstyle=\color{mediumgray}\upshape,
	emph={},
	emphstyle=\color{crimson},
	extendedchars=true,  % requires inputenc
	fontadjust=true,
	frame=single,
	identifierstyle=\color{black},
	keepspaces=true,
	keywordstyle=\color{mediumblue},
	keywordstyle={[2]\color{darkviolet}},
	keywordstyle={[3]\color{royalblue}},
	literate=%
	{á}{{\'a}}1 {č}{{\v{c}}}1 {ď}{{\v{d}}}1 {é}{{\'e}}1 {ě}{{\v{e}}}1
	{í}{{\'i}}1 {ň}{{\v{n}}}1 {ó}{{\'o}}1 {ř}{{\v{r}}}1 {š}{{\v{s}}}1
	{ť}{{\v{t}}}1 {ú}{{\'u}}1 {ů}{{\r{u}}}1 {ý}{{\'y}}1 {ž}{{\v{z}}}1,		
	numbers=left,
	numbersep=5pt,
	numberstyle=\tiny\color{black},
	rulecolor=\color{black},
	showlines=true,
	showspaces=false,
	showstringspaces=false,
	showtabs=false,
	stringstyle=\color{forestgreen},
	tabsize=2,
	title=\lstname,
	upquote=true  % requires textcomp	
}
% copied code end






%% Bordel pro práci - můžeš smáznout :) 
%%%%%%%%%%%%%%%%%%%

\usepackage{lipsum} %% balíček který píše lipsum (nesmyslný text, který se používá pro kontrolu typografie)

%% Začátek dokumentu
%%%%%%%%%%%%%%%%%%%%
\begin{document}
	
	\pagestyle{empty}
	\pagenumbering{Roman}
	
	\cleardoublepage

%% Titulní stránka
%%%%%%%%%%%%%%%%%%%%%%%%%%%%%%%%%%%%%%%%
 
{\fontfamily{phv}\selectfont

\begin{center}
    \includegraphics[width=0.5\linewidth]{spskola.png}

    \vspace{0.1cm} 

    {\bfseries
        \LARGE ZÁVĚREČNÁ STUDIJNÍ PRÁCE\\
        \large dokumentace\\
        \vspace{0.3cm}
        \LARGE \nazevPrace\\[0.3cm]
        \includegraphics[width=0.4\textwidth]{logo-fretfly.png}
    }
\end{center}

}

	\begin{table}[H]
		\begin{tabular}{ll}
			\textbf{Autor:} & \jmenoAutora\\ 
			\textbf{Obor:} & \kodOboru { } \obor\\
			\textbf{} & \zamereni\\
			\textbf{Třída:} & \trida\\
			\textbf{Školní rok:} & \skolniRok\\
		\end{tabular}
	\end{table}		
}
 
\newpage
\cleardoublepage %% Zalomení dvojstránky
	
%% Stránka obsahující poděkování 
%%%%%%%%%%%%%%%%%%%%%%%%%%%%%%%%%%%%%%%%%%%%%%%%%%%%%%%%

%% Poděkování - nepovinné
%%%%%%%%%%%%%%%%%%%%%%%%%%%%
	

	
	\vspace*{0.7\textheight} %% Vertikální mezeru je možné upravit

%% Prohlášení - povinné
%%%%%%%%%%%%%%%%%%%%%%%%%%%%
	\noindent{\large{\bfseries{Prohlášení}\\}}  %% uprav si koncovky podle toho na jaký rod se cítíš, vypadá to pak lépe :) 
	\noindent{Prohlašuji, že jsem závěrečnou práci vypracoval samostatně a uvedl veškeré použité 
		informační zdroje.\\}
	\noindent{Souhlasím, aby tato studijní práce byla použita k výukovým a prezentačním účelům na Střední průmyslové a umělecké škole v Opavě, Praskova 399/8.}
	\vfill
	\noindent{V Opavě \datumOdevzdani\\}
	\noindent
	\begin{minipage}{\linewidth}
		\hspace{9.5cm} 
		\begin{tabular}{@{}p{6cm}@{}}
			\dotfill \\
			Podpis autora
		\end{tabular}
	\end{minipage}
	
	\cleardoublepage %% Zalomení dvojstránky

%% Stránka obsahující abstrakt (anotaci)
%%%%%%%%%%%%%%%%%%%%%%%%%%%%%%%%%%%%%%%%%%%%%%%%%%%%%%%%	

%% Abstrakt v češtině
%%%%%%%%%%%%%%%%%%%%%%%%%%%%
	\noindent{\Large{\bfseries{Abstrakt}\\}}
Práce se zabývá tvorbou mobilní aplikace pro kytaristy s~nástroji pro učení a cvičení. Součástí práce byla implementace ladičky s~detekcí frekvence, metronomu s~nastavitelným tempem a~databáze kytarových akordů s vizualizací hmatníku. Výsledkem je funkční mobilní aplikace vytvořená ve Flutteru, která umožňuje kytaristům ladit nástroj pomocí mikrofonu, cvičit rytmus s~metronomem a procházet rozsáhlou databázi akordů s diagramy hmatníku. Aplikace obsahuje autentizační systém s podporou přihlášení přes email, uživatelský profil se statistikami pokroku včetně sledování naučených akordů a~počtu dní v řadě. Mezi hlavní funkce patří ladička s~analýzou frekvence pomocí zero-crossing detection, metronom s nastavitelným BPM (40–240) a~taktem, databáze akordů s vyhledáváním a filtrováním podle základního tónu, a~systém pro označování naučených akordů. Aplikace využívá Firebase pro autentizaci a~ukládání dat v~Cloud Firestore, podporuje dark mode a~light mode.


	
	\vspace{18pt}
	
	\noindent{\large{\bfseries{Klíčová slova}}}
	
	\noindent mobilní aplikace, Flutter, Firebase, kytara
	
	\vspace{18pt}

%% Abstrakt v angličtině
%%%%%%%%%%%%%%%%%%%%%%%%%%%%	
	\noindent{\Large{\bfseries{Abstract}}}
	
This thesis focuses on the development of a mobile application for guitarists that provides tools for learning and practice. The work included the implementation of a tuner with frequency detection, a metronome with adjustable tempo, and a database of guitar chords with fretboard visualization. The result is a fully functional mobile application developed in Flutter that allows guitarists to tune their instrument using the device microphone, practice rhythm with a~metronome, and browse an extensive chord database with fretboard diagrams.

The application includes an authentication system with support for email sign-in, a user profile with progress statistics, including tracking learned chords and consecutive practice days. The main features include a tuner with frequency analysis using zero-crossing detection, a~metronome with adjustable BPM (40–240) and time signature, a~chord database with search and filtering by root note, and a system for marking learned chords. The application uses Firebase for authentication and data storage in Cloud Firestore and supports both dark and light modes.
	
	\vspace{18pt}
	
	\noindent{\large{\bfseries{Keywords}}}
	
	\noindent mobile application, Flutter, Firebase, guitar
	
	\cleardoublepage %% Zalomení stránky

%% Stránka s generovaným obsahem
%%%%%%%%%%%%%%%%%%%%%%%%%%%%%%%%%%%%%%%	
	\setcounter{tocdepth}{2}
	\tableofcontents %% Vygeneruje tabulku s obsahem

	\pagenumbering{arabic} %% Nastavení způsobu číslování stránek (alternativy roman | Roman)
	\setcounter{page}{1} %% Nastavení počitadla stránek

%% Stránka s úvodem - povinná část
%%%%%%%%%%%%%%%%%%%%%%%%%%%%%%%%%%%%%%%		
	\chapter*{Úvod}
	\addcontentsline{toc}{chapter}{Úvod}
Mobilní~aplikace určené pro výuku hry na~hudební nástroje jsou dnes běžně využívány začínajícími i~pokročilými hudebníky. Ani já nejsem výjimkou, a~proto bylo cílem této práce vytvořit mobilní~aplikaci, která nabídne základní nástroje, jež jsou v~současnosti nedílnou součástí hry na~kytaru.

Téma bylo zvoleno z~důvodu osobního zájmu o~hru na~kytaru a~zároveň snahy vytvořit praktickou~aplikaci s~reálným využitím. Projekt byl realizován ve~frameworku Flutter a~využívá platformu Firebase pro správu uživatelských dat a~službu Cloud~Firestore pro ukládání databáze kytarových akordů.

Hlavním cílem práce bylo vytvořit mobilní~aplikaci FretFly, která nabídne základní nástroje pro efektivní cvičení na~kytaru. Mezi tyto nástroje patří ladička využívající mikrofon mobilního~zařízení, metronom s~nastavitelným tempem a~databáze kytarových akordů s~přehledným zobrazením prstokladu na~hmatníku. Součástí aplikace je také uživatelský~účet, který umožňuje ukládat informace o~pokroku, například naučené akordy nebo počet dní cvičení v~řadě.


\chapter{Flutter a mobilní vývoj}

\section{Co je to Flutter}
\label{sec:co_je_to_flutter}
Flutter je open-source~framework od~společnosti Google určený pro vývoj multiplatformních~aplikací. Umožňuje vytvářet mobilní~aplikace pro operační systémy Android a~iOS z~jedné společné kódové základny, což zjednodušuje vývoj i~následnou údržbu aplikace. Framework využívá programovací jazyk Dart a~je založen na~vlastním vykreslovacím enginu, který zajišťuje vysoký výkon a~plynulé uživatelské rozhraní.

Flutter poskytuje rozsáhlou~sadu předpřipravených grafických komponent (widgetů), díky nimž je možné rychle vytvářet moderní a~responzivní uživatelská rozhraní. Výhodou je také funkce hot~reload, která umožňuje okamžitě zobrazit změny v~kódu bez nutnosti znovu spouštět celou~aplikaci, což výrazně urychluje vývoj.



\section{Výhody Flutter frameworku}
\label{sec:vyuziti_AR}


\begin{itemize}
	\item vývoj aplikací pro Android i iOS z jedné společné kódové základny,
	\item vysoký výkon díky vlastnímu vykreslovacímu enginu,
	\item moderní a konzistentní uživatelské rozhraní napříč platformami,
	\item otevřený (open-source) framework s aktivní komunitou,
	\item dobrá integrace s dalšími službami, například Firebase.
\end{itemize}


\section{Programovací jazyk Dart}
\label{sec:programovaci_jazyk_dart}


\subsection{Co je to Dart}
Dart je moderní, objektově orientovaný programovací jazyk vyvinutý společností Google. Používá se především pro vývoj~aplikací ve~frameworku Flutter. Dart je navržen tak, aby byl jednoduchý na~čtení a~psaní, podporuje statické typování, asynchronní programování a~nabízí vysoký výkon díky kompilaci do~nativního kódu pro mobilní~i~webové~aplikace.


\paragraph{Výhody}
\begin{itemize}
	\item Snadná integrace s Flutterem pro vývoj multiplatformních aplikací,
	\item Podpora statického typování a objektově orientovaného programování, což zvyšuje bezpečnost a přehlednost kódu,
	\item Možnost kompilace do nativního kódu pro vysoký výkon aplikací.
\end{itemize}

\paragraph{Nevýhody}
\begin{itemize}
	\item Menší komunita a méně dostupných knihoven než u populárnějších jazyků, například JavaScriptu,
	\item Relativně méně zkušených vývojářů, což může komplikovat hledání odborné podpory,
	\item Méně rozšířený mimo Flutter, takže se hodí spíše pro specifické projekty než pro obecný vývoj softwaru.
\end{itemize}

\subsection{Použití Dart ve Flutteru}
Dart je primárním programovacím jazykem frameworku Flutter a~je klíčový pro tvorbu~aplikací v~tomto prostředí. Ve~Flutteru slouží Dart k~definici widgetů, což jsou základní stavební bloky uživatelského rozhraní. Každý vizuální prvek~aplikace, jako jsou tlačítka, texty nebo obrázky, je reprezentován widgetem, jehož chování a~vzhled je definován v~Dartu.

Dart rovněž umožňuje efektivní práci s~asynchronními úlohami, což je při vývoji mobilních~aplikací důležité například při načítání dat z~databáze nebo z~internetu. Díky hot~reload ve~Flutteru lze změny v~Dart~kódu okamžitě zobrazit v~aplikaci, což výrazně urychluje vývoj a~testování.

Použití Dartu ve~Flutteru přináší jednotnou a~přehlednou strukturu~kódu, snadnější údržbu a~možnost vytvářet multiplatformní~aplikace pro Android i~iOS z~jedné kódové základny. Jazyk také podporuje moderní programovací principy, jako je objektově orientované programování, statické typování a~modularita, které usnadňují tvorbu složitějších a~dobře udržovatelných~aplikací.


	

\chapter {Funkce mobilní aplikace FretFly}

\section {Metronom}
\label{metronom}
Metronom je nástroj pro udržování konstantního~tempa při cvičení na~kytaru. Pomáhá držet správné~tempo, a~zároveň je to skvělý pomocník pro trénink~rytmu. Při vytváření jsem měl trošku problém s~implementací zvukového "tiku", který měl znázorňovat jednu~dobu. Nakonec ale metronom funguje vcelku slušně, a~ikdyž to není ještě úplně stoprocentní, svůj účel držení~tempa splňuje.


\subsection {Účel a funkce metronomu}
Metronom v aplikaci \textbf{FretFly} poskytuje následující funkce:

\begin{itemize}
    \item \textbf{Nastavení tempa v rozsahu 40--240 BPM} \\
    Uživatel si může zvolit tempo podle náročnosti cvičení, od pomalého procvičování až po rychlá rytmická cvičení.

    \item \textbf{Výběr taktu (2/4, 3/4, 4/4, 6/4)} \\
    Metronom podporuje více taktových označení, což umožňuje přizpůsobení různým hudebním stylům a skladbám.

    \item \textbf{Zvukový výstup s odlišením první doby} \\
    První doba v taktu je zvýrazněna odlišným zvukem, což usnadňuje orientaci v rytmu.

    \item \textbf{Vizuální indikátor aktuální doby} \\
    Grafický prvek zobrazuje průběh taktu a aktuální dobu, což je užitečné při cvičení bez zvuku nebo v hlučném prostředí.

    \item \textbf{Tap tempo pro nastavení BPM poklepáním} \\
    Uživatel může nastavit tempo opakovaným poklepáním na obrazovku podle rytmu konkrétní skladby.
\end{itemize}


\subsection{Technická implementace}

Metronom je v aplikaci \textbf{FretFly} implementován jako \texttt{StatefulWidget} s názvem
\texttt{MetronomePage}, který spravuje svůj stav pomocí několika klíčových proměnných. Tyto
proměnné určují chování metronomu, jeho aktuální nastavení a průběh přehrávání.

\begin{itemize}
    \item \textbf{\_isPlaying} \\
    Logická proměnná indikující, zda je metronom aktuálně spuštěn, nebo zastaven.

    \item \textbf{\_bpm} \\
    Proměnná uchovávající aktuální tempo metronomu v úderech za minutu (BPM), přičemž
    podporovaný rozsah je 40 až 240 BPM.

    \item \textbf{\_beatsPerBar} \\
    Určuje počet dob v jednom taktu. Hodnoty odpovídají podporovaným taktům
    (2, 3, 4 nebo 6).

    \item \textbf{\_currentBeat} \\
    Sleduje aktuální dobu v rámci taktu, přičemž nabývá hodnot od 1 do hodnoty
    \texttt{\_beatsPerBar}.

    \item \textbf{\_timer} \\
    Periodický časovač, který zajišťuje pravidelné spouštění jednotlivých dob metronomu
    na základě nastaveného tempa.
\end{itemize}


\subsection{Výpočet intervalu mezi dobami}

Interval mezi jednotlivými dobami metronomu je odvozen z aktuálně nastaveného tempa,
které je udáváno v jednotkách BPM (beats per minute). Jelikož jedna minuta odpovídá
60\,000 milisekundám, lze interval mezi jednotlivými dobami vypočítat podle následujícího
vzorce:

\[
\text{interval} = \frac{60000}{\text{BPM}} \; \text{ms}
\]

Například při nastaveném tempu 120 BPM vychází interval mezi jednotlivými dobami na
500~ms, tedy 0{,}5 sekundy. Tento interval určuje, jak často je spuštěna další doba
metronomu.

V aplikaci je výpočet intervalu a spuštění metronomu realizováno v metodě
\texttt{\_startMetronome()}, která nejprve okamžitě přehraje první dobu a následně
nastaví periodický časovač pro další doby. Implementace této metody je uvedena níže:

\begin{lstlisting}[style=csh, caption={Ukázka kódu pro spuštění metronomu.}]
void _startMetronome() {
  _playBeat();

  final interval = Duration(
    milliseconds: (60000 / _bpm).round(),
  );

  _timer = Timer.periodic(interval, (timer) {
    _playBeat();
  });
}
\end{lstlisting}

Metoda \texttt{\_startMetronome()} nejprve zavolá funkci \texttt{\_playBeat()}, čímž zajistí
okamžité přehrání první doby bez zpoždění. Následně je vypočítán časový interval na základě
aktuální hodnoty BPM a vytvořen objekt třídy \texttt{Duration}. Tento interval je následně
použit při vytvoření periodického časovače \texttt{Timer.periodic}, který v pravidelných
časových intervalech spouští další doby metronomu.

Tímto způsobem je zajištěno přesné a stabilní časování jednotlivých úderů metronomu
po celou dobu jeho běhu.

\subsection{Zvukový výstup}

Zvukový výstup metronomu je realizován pomocí balíčku \texttt{just\_audio}, který umožňuje
přesné a nízkolatenční přehrávání zvukových souborů. Pro zajištění jasného rozlišení první
doby v taktu jsou použity dvě samostatné instance třídy \texttt{AudioPlayer}.

\begin{itemize}
    \item \textbf{\_accentPlayer} \\
    Slouží k přehrávání první doby v taktu (akcentované doby).

    \item \textbf{\_regularPlayer} \\
    Používá se pro přehrávání ostatních, neakcentovaných dob.
\end{itemize}

První doba v taktu je zvukově i hapticky zvýrazněna, aby byla pro uživatele snadno
rozpoznatelná. Toto zvýraznění je realizováno kombinací změny parametrů přehrávání
a silnější haptické odezvy:

\begin{itemize}
    \item přehrávání se zvýšenou rychlostí (\texttt{speed = 2.0}), čímž dojde ke zvýšení
    výšky tónu,
    \item přehrávání s plnou hlasitostí (\texttt{volume = 1.0}),
    \item silnější haptická odezva pomocí metody
    \texttt{HapticFeedback.mediumImpact()}.
\end{itemize}

Ostatní doby jsou přehrávány s méně výrazným zvukovým i haptickým projevem, aby bylo
zachováno jasné rozlišení akcentované doby:

\begin{itemize}
    \item přehrávání při standardní rychlosti (\texttt{speed = 1.0}),
    \item snížená hlasitost (\texttt{volume = 0.7}),
    \item lehká haptická odezva pomocí metody
    \texttt{HapticFeedback.lightImpact()}.
\end{itemize}

Zvukový soubor metronomu je uložen v adresáři
\texttt{assets/sounds/sound-effect-hd.mp3} a je načten při inicializaci widgetu
\texttt{MetronomePage}. Tím je zajištěno, že zvuk je připraven k okamžitému přehrání bez
zbytečného zpoždění během běhu metronomu.

Použití oddělených přehrávačů pro akcentované a neakcentované doby umožňuje nezávislé
nastavení parametrů přehrávání a přispívá k lepší čitelnosti rytmu a celkové uživatelské
zkušenosti při cvičení.


\subsection{Tap Tempo funkce}

Funkce Tap Tempo umožňuje uživateli nastavit tempo metronomu (BPM) manuálně
poklepáním na tlačítko v požadovaném rytmu. Tato metoda je vhodná zejména v situacích,
kdy uživatel nezná přesnou hodnotu BPM, ale dokáže tempo intuitivně zadat.

Algoritmus Tap Tempo pracuje s časovými značkami jednotlivých poklepů a skládá se
z následujících kroků:

\begin{enumerate}
    \item \textbf{Ukládání časových značek} \\
    Při každém poklepu je aktuální čas uložen do seznamu \texttt{\_tapTimes}.
    Algoritmus uchovává maximálně čtyři poslední poklepy, přičemž starší hodnoty
    jsou postupně odstraňovány.

    \item \textbf{Výpočet intervalů mezi poklepy} \\
    Z uložených časových značek jsou vypočítány časové rozdíly mezi jednotlivými
    poklepy v milisekundách.

    \item \textbf{Výpočet průměrného intervalu} \\
    Z vypočtených intervalů je určen průměrný interval, který slouží ke stabilizaci
    výsledného tempa a omezení vlivu nepřesného poklepu.

    \item \textbf{Přepočet na BPM} \\
    Průměrný interval je následně převeden na hodnotu BPM podle vztahu:
    
    \[
    BPM = \frac{60000}{\text{průměrný interval v ms}}
    \]

    \item \textbf{Omezení výsledné hodnoty} \\
    Vypočtená hodnota BPM je omezena na povolený rozsah 40–240 BPM, aby odpovídala
    funkčním limitům metronomu.
\end{enumerate}

Pro zajištění správné funkce je implementováno také časové omezení. Pokud uživatel
neprovede další poklep po dobu delší než dvě sekundy, seznam časových značek
\texttt{\_tapTimes} je automaticky vymazán a nový výpočet začíná od začátku.

Vlastní implementace Tap Tempo je realizována metodou \texttt{\_tapTempo()}, jejíž
zdrojový kód je uveden níže:

\begin{lstlisting}[style=csh, caption={Ukázka kódu pro funkci Tap Tempo.}]
void _tapTempo() {
  final now = DateTime.now();
  _tapTimes.add(now);

  if (_tapTimes.length > 4) {
    _tapTimes.removeAt(0);
  }

  if (_tapTimes.length >= 2) {
    final intervals = <int>[];

    for (int i = 1; i < _tapTimes.length; i++) {
      intervals.add(
        _tapTimes[i]
            .difference(_tapTimes[i - 1])
            .inMilliseconds,
      );
    }

    final avgInterval =
        intervals.reduce((a, b) => a + b) / intervals.length;

    final newBpm =
        (60000 / avgInterval).round().clamp(40, 240);

    setState(() {
      _bpm = newBpm;

      if (_isPlaying) {
        _stopMetronome();
        _startMetronome();
      }
    });
  }
}
\end{lstlisting}


Po změně hodnoty BPM je metronom v případě aktivního přehrávání automaticky restartován,
aby se nové tempo okamžitě projevilo v běhu aplikace.

\begin{figure}[h]
  \centering
  \includegraphics[width=0.4\textwidth]{metronom.png}
  \caption{Ukázka rozhraní Metronomu.}
  \label{fig:signup_page}
\end{figure}

\clearpage

\section {Ladička}
\label{ladicka}
Ladička je nástroj, který dokáže přesně naladit každou~strunu bez větší~námahy. Na trhu už jsou nějakou dobu klasické ladičky - malá plastová~zařízení s~analogovým, ale už i~s~digitálním~displejem, které se na~kytaru fyzicky připnou a~naladí ji. Myslím si ale, že v~dnešní době už drtivá většina muzikantů (minimálně těch amatérských) zvolí levnější~variantu a~to ladičku v~mobilu. Já sám jsem nikdy fyzickou~ladičku nevlastnil a~neměl jsem s~mobilní~verzí sebemenší problém. Proto v~mé~aplikaci nemohla chybět.


\subsection{Standardní ladění kytary}

Standardní ladění kytary je označováno jako \textbf{E-A-D-G-B-E} a představuje nejběžnější způsob ladění pro šestistrunnou kytaru. Každá struna má přesně definovanou frekvenci, která slouží jako cílový tón při ladění:

\begin{center}
\begin{tabular}{|c|c|c|}
\hline
\textbf{Struna} & \textbf{Pořadí} & \textbf{Frekvence [Hz]} \\
\hline
E6 & 6 (nejtlustší) & 82,41 \\
A5 & 5 & 110,00 \\
D4 & 4 & 146,83 \\
G3 & 3 & 196,00 \\
B2 & 2 & 246,94 \\
E1 & 1 (nejtenčí) & 329,63 \\
\hline
\end{tabular}
\end{center}

Tyto frekvence jsou v aplikaci uloženy v konstantě \_stringFrequencies a slouží pro porovnání s detekovanou frekvencí během ladění:

\begin{verbatim}
static const Map<String, double> _stringFrequencies = {
  'E6': 82.41,
  'A5': 110.00,
  'D4': 146.83,
  'G3': 196.00,
  'B2': 246.94,
  'E1': 329.63
};
\end{verbatim}

Digitální ladička porovnává detekovaný tón s touto referenční hodnotou a poskytuje okamžitou vizuální zpětnou vazbu uživateli, což umožňuje přesné a rychlé ladění.


\subsection{Technická implementace}

Ladička v aplikaci je implementována jako interaktivní stránka (\textit{StatefulWidget} TunerPage), která spravuje stav nahrávání a analýzy zvuku. Hlavními prvky jsou:

\begin{itemize}
    \item \textbf{AudioRecorder} – nástroj pro záznam zvuku z mikrofonu.
    \item \textbf{Indikátor nahrávání} – ukazuje, zda aplikace právě nahrává.
    \item \textbf{Detekovaný tón a frekvence} – aktuálně zaznamenaný tón (E, A, D, G, B) a jeho frekvence v Hz.
    \item \textbf{Odchylka od cílového tónu} – vyjádřená v centech, ukazuje, o kolik je struna mimo ladění.
    \item \textbf{Vybraná struna} – možnost zaměřit se na konkrétní strunu.
\end{itemize}

\subsubsection*{Ukázka implementace}

\begin{lstlisting}[style=csh, caption={Ukázka kódu inicializace ladičky a jejích parametrů.}]
// Inicializace hlavních proměnných
final AudioRecorder _recorder = AudioRecorder();
bool _isRecording = false;
String? _detectedNote;
double _detectedFrequency = 0.0;
double _deviation = 0.0; // odchylka v centech
String? _selectedString;
Timer? _analysisTimer;

// Technické parametry zpracování zvuku
static const int sampleRate = 44100;
static const int bufferSize = 4096;
\end{lstlisting}


\noindent
Tento kód ukazuje, jak aplikace uchovává informace o nahrávání a analyzovaném zvuku. Každých 200 ms probíhá analýza nahrávky, která zjišťuje aktuální frekvenci a odchylku od cílového tónu. Díky tomu uživatel okamžitě vidí, zda je struna naladěna správně, nebo zda je potřeba ji napnout či uvolnit.

\subsubsection*{Technické parametry}
\begin{itemize}
    \item \textbf{Vzorkovací frekvence:} 44100 Hz.
    \item \textbf{Velikost bufferu:} 4096 vzorků.
    \item \textbf{Formát:} PCM16 (16-bit signed integer).
    \item \textbf{Interval analýzy:} každých 200 ms.
\end{itemize}

Tyto parametry zajišťují přesné a rychlé zachycení frekvence každé struny a umožňují okamžitou vizuální zpětnou vazbu uživateli.

\subsection{Záznam zvuku a detekce frekvence}

Záznam zvuku v aplikaci probíhá pomocí třídy \texttt{AudioRecorder}, která zachycuje signál z mikrofonu. Nahrávání je nastaveno s parametry:

\begin{itemize}
    \item \textbf{Formát:} PCM16 (16-bit signed integer).
    \item \textbf{Vzorkovací frekvence:} 44100 Hz.
    \item \textbf{Kanály:} mono (jediný kanál).
\end{itemize}

Periodická analýza probíhá každých 200 ms, aby aplikace mohla průběžně detekovat aktuální tón. Ukázka inicializace nahrávání v kódu:

\begin{lstlisting}[style=csh, caption={Ukázka kódu spuštění záznamu zvuku.}]
await _recorder.start(
  const RecordConfig(
    encoder: AudioEncoder.pcm16bits,
    sampleRate: sampleRate,
    numChannels: 1,
  ),
  path: _recordingPath!,
);

// Periodická analýza zvuku každých 200 ms
_analysisTimer = Timer.periodic(
  const Duration(milliseconds: 200),
  (timer) {
    _analyzeAudioFile();
  },
);
\end{lstlisting}


Detekce frekvence využívá princip \textbf{zero-crossing detection}, což je metoda, která počítá, kolikrát signál prochází nulovou hodnotou. Z tohoto počtu a délky nahrávky lze vypočítat frekvenci tónu. Přehled hlavních kroků:

\begin{itemize}
    \item Normalizace vzorků – odstranění posunu signálu (DC offset).
    \item Počítání průchodů nulou s filtrováním šumu (threshold).
    \item Výpočet frekvence: počet průchodů nulou dělený délkou vzorku.
    \item Filtrování nereálných frekvencí – pro kytaru platí rozsah 50–500 Hz.
\end{itemize}

Ukázka funkce pro detekci frekvence:

\begin{lstlisting}[style=csh, caption={Ukázka kódu detekce frekvence metodou průchodu nulou.}]
double _detectFrequency(List<int> samples) {
  final normalized = _normalizeSamples(samples);
  const threshold = 100;
  int zeroCrossings = 0;
  bool wasPositive = normalized[0] > 0;

  for (int i = 1; i < normalized.length; i++) {
    final current = normalized[i];
    if (wasPositive && current < -threshold) {
      zeroCrossings++;
      wasPositive = false;
    } else if (!wasPositive && current > threshold) {
      zeroCrossings++;
      wasPositive = true;
    }
  }

  final duration = samples.length / sampleRate;
  final frequency = (zeroCrossings / 2) / duration;

  if (frequency < 50 || frequency > 500) {
    return 0.0;
  }

  return frequency;
}
\end{lstlisting}


Tento princip umožňuje aplikaci rozpoznat aktuálně hraný tón a jeho frekvenci, což je základ pro následné vyhodnocení odchylky a vizuální indikaci ladění.


\subsection{Výpočet odchylky a vizuální indikátory}

Po detekci frekvence hraného tónu aplikace vypočítá odchylku od cílového tónu v \textbf{centech} (1 cent = 1/100 půltónu). Tento způsob měření umožňuje přesně vyjádřit, jak moc je tón nad nebo pod ideálním laděním.

\begin{lstlisting}[style=csh, caption={Ukázka kódu pro výpočet odchylky tónu v centech.}]
double _calculateDeviation(double detectedFreq, double targetFreq) {
  if (detectedFreq <= 0 || targetFreq <= 0) {
    return 0.0;
  }

  final cents =
      1200 * (log(detectedFreq / targetFreq) / log(2));

  return cents;
}
\end{lstlisting}

Matematicky lze odchylku vyjádřit vzorcem:

\[
\text{cents} = 1200 \times \frac{\log(f_{\text{detekovaná}} / f_{\text{cílová}})}{\log(2)}
\]

Pro intuitivní vizuální zpětnou vazbu aplikace používá barevné indikátory:

\begin{itemize}
    \item \textbf{Zelená:} odchylka menší než ±5 centů – tón je perfektně naladěn.
    \item \textbf{Oranžová:} odchylka mezi ±5 a ±20 centy – tón je blízko ideálu.
    \item \textbf{Červená:} odchylka větší než ±20 centů – tón je výrazně mimo ladění.
\end{itemize}

Aplikace zobrazuje:

\begin{itemize}
    \item Detekovaný tón a jeho frekvenci v Hz
    \item Vizuální indikátor ladění s posuvníkem
    \item Barevný indikátor odchylky (zelená, oranžová, červená)
    \item Textové instrukce pro ladění – napnout nebo uvolnit strunu
\end{itemize}

Tento přístup zajišťuje, že uživatel okamžitě vidí, zda je struna správně naladěna, a dostává jasné instrukce, jak postupovat.
\begin{figure}[h]
  \centering
  \includegraphics[width=0.4\textwidth]{ladicka.png}
  \caption{Ukázka rozhraní Ladičky.}
  \label{fig:signup_page}
\end{figure}

\section{Databáze akordů}
\label{sec:databaze_akordu}
Myslím si, že trošku méně atraktivnější část~aplikace, ale rozhodně nezbytná. Mít akordy na~jednom uspořádaném~místě je určitě příjemné. Proto jsem chtěl jednu takovou alespoň s~pár~akordy i~v~mé~aplikaci. Na začátku jsem plánoval, že udělám databázi lokální, ale jelikož jsem se rozhodl použít Firebase pro ukládání přihlášených~uživatelů, použil jsem Firebase i~pro vytvoření modelu pro databázi~akordů.


\newpage
\subsection{Model akordu}

Akord je v aplikaci reprezentován třídou \texttt{Chord}, která obsahuje následující vlastnosti:

\begin{lstlisting}[style=csh, caption={Ukázka třídy pro reprezentaci akordu.}]
class Chord {
  final String id;
  final String name;
  final String category;
  final List<int> fingering;   // [0, 3, 2, 0, 1, 0] pro struny E-A-D-G-B-E
  final int position;          // pozice na hmatníku (1 = první pražec)
  final bool hasBarre;
  final int? barreFret;
  final String? description;
  final List<String> tags;
  final String difficulty;     // "beginner", "intermediate", "advanced"
}
\end{lstlisting}


\noindent
Význam jednotlivých vlastností:

\begin{itemize}
  \item \textbf{id}: unikátní identifikátor akordu.
  \item \textbf{name}: název akordu (např. C, G7, Am).
  \item \textbf{category}: kategorie akordu (např. dur, moll, sedmý).
  \item \textbf{fingering}: pole pozic prstů pro jednotlivé struny (E-A-D-G-B-E):
    \begin{itemize}
      \item \texttt{0} = otevřená struna
      \item \texttt{-1} = neznělá struna
      \item \texttt{1-24} = pozice prstu na pražci
    \end{itemize}
  \item \textbf{position}: pozice akordu na hmatníku (1 = první pražec).
  \item \textbf{hasBarre}: zda akord obsahuje barré.
  \item \textbf{barreFret}: pokud je barré, na kterém pražci.(Barré je kytarová technika, ve které ukazováček stiskne více strun najednou)
  \item \textbf{description}: volitelný text s vysvětlením akordu.
  \item \textbf{tags}: seznam tagů pro vyhledávání nebo filtrování akordů.
  \item \textbf{difficulty}: obtížnost akordu – \texttt{"beginner"}, \texttt{"intermediate"} nebo \texttt{"advanced"}.
\end{itemize}


\subsection{ChordsService}

\textit{ChordsService} je specializovaná služba poskytující jednotný přístup k databázi akordů umístěné v \textit{Cloud Firestore}. Tato služba centralizuje operace s akordy, což umožňuje snadné načítání, vyhledávání a filtrování akordů napříč aplikací. Použití služby zajišťuje, že všechny komponenty aplikace pracují s konzistentními daty a snižuje duplicitu kódu.

Hlavní funkce \textit{ChordsService} zahrnují:

\begin{itemize}
    \item Načtení kompletního seznamu akordů.
    \item Vyhledávání akordů podle názvu.
    \item Filtrování akordů podle kategorie (např. mollové, durové).
    \item Filtrování akordů podle obtížnosti (začátečník, středně pokročilý, pokročilý).
\end{itemize}

\begin{lstlisting}[style=csh, caption={Ukázka implementace služby ChordsService.}]
class ChordsService {
  static final ChordsService _instance = ChordsService._internal();
  factory ChordsService() => _instance;

  final FirebaseFirestore _firestore = FirebaseFirestore.instance;
  static const String _collection = 'chords';

  // Načte všechny akordy z kolekce
  Stream<List<Chord>> getAllChords() {
    return _firestore
        .collection(_collection)
        .orderBy('name')
        .snapshots()
        .map((snapshot) => snapshot.docs
            .map((doc) => Chord.fromMap(doc.data(), doc.id))
            .toList());
  }

  // Vyhledávání akordů podle názvu
  Stream<List<Chord>> searchChords(String query) {
    if (query.isEmpty) return getAllChords();

    return _firestore
        .collection(_collection)
        .where('name', isGreaterThanOrEqualTo: query)
        .where('name', isLessThan: '${query}z')
        .snapshots()
        .map((snapshot) => snapshot.docs
            .map((doc) => Chord.fromMap(doc.data(), doc.id))
            .toList());
  }
}
\end{lstlisting}

\subsection{Inicializace databáze}

Při prvním spuštění aplikace je nezbytné inicializovat databázi akordů s předdefinovanou sadou základních akordů. Tento proces zajistí, že aplikace má ihned k dispozici všechny potřebné akordy, které mohou být dále načítány a filtrovány pomocí služby \textit{ChordsService}.

Databáze obsahuje akordy z kategorií \textit{Major}, \textit{Minor} a \textit{Seventh} s různými úrovněmi obtížnosti, což umožňuje uživatelům od začátečníků po pokročilé pohodlné používání aplikace.

\begin{lstlisting}[style=csh, caption={Ukázka inicializace databáze akordů.}]
Future<void> initializeChords() async {
  final chords = [
    Chord(
      id: 'c_major',
      name: 'C',
      category: 'Major',
      fingering: [0, 1, 0, 2, 1, 0],
      position: 1,
      tags: ['major', 'open', 'beginner'],
      difficulty: 'beginner',
    ),
    // ... další akordy
  ];

  for (final chord in chords) {
    final docRef = _firestore.collection(_collection).doc(chord.id);
    final existing = await docRef.get();
    if (!existing.exists) {
      await docRef.set(chord.toMap());
    }
  }
}
\end{lstlisting}

\textbf{Poznámka:} Tento postup zajišťuje, že každý akord je v databázi pouze jednou a nedochází k duplicitám. Inicializace podporuje snadnou rozšiřitelnost – nové akordy mohou být přidány do seznamu a automaticky synchronizovány s Firestore.
\begin{figure}[h]
  \centering
  \includegraphics[width=0.4\textwidth]{databaze_akordu.png}
  \caption{Ukázka rozhraní databáze s akordy.}
  \label{fig:signup_page}
\end{figure}

\chapter{Přihlašovací systém}
\label{funkce_aplikace}
Jak už jsem zmínil u~kapitoly 2.3 Databáze~akordů, pro ukládání~uživatelů jsem použil Firebase~Authentication. Původně jsem zamýšlel, že zprovozním i~přihlášení přes Google a~Apple, nakonec jsem od toho hlavně kvůli nedostatku~času upustil, a~tak je v~aplikaci pouze přihlášení přes e-mail. Do budoucna bych však přihlášení přes Google a~Apple moc rád do~aplikace přidal.


\section{AuthService}

\textit{AuthService} je singletonová služba, která centralizuje veškeré operace související s autentizací uživatelů v aplikaci. Zajišťuje bezpečné přihlášení, registraci a správu přihlášeného uživatele.

Služba obsahuje následující funkcionality:
\begin{itemize}
  \item Přihlášení pomocí e-mailu a hesla
  \item Registrace nového uživatele
  \item Odhlášení uživatele
\end{itemize}

\begin{lstlisting}[style=csh, caption={Ukázka implementace přihlášení přes e-mail a heslo.}]
class AuthService {
  AuthService._();
  static final AuthService instance = AuthService._();

  Future<void> signIn({
    required String email,
    required String password
  }) async {
    try {
      await FirebaseAuth.instance.signInWithEmailAndPassword(
        email: email.trim(),
        password: password,
      );
      
      await StreakService().updateOnLogin();

      final p = await _prefs;
      await p.setString(
        _keyCurrentUser,
        FirebaseAuth.instance.currentUser?.email ?? '',
      );
    } on FirebaseAuthException catch (e) {
      throw AuthException(_mapFirebaseError(e));
    }
  }
}
\end{lstlisting}

\textbf{Poznámka:} Centralizace autentizace v \textit{AuthService} zajišťuje jednotnou logiku přihlášení a snadnou správu chyb napříč celou aplikací.

\section{LoginPage}

\textit{LoginPage} poskytuje uživateli rozhraní pro přihlášení do aplikace s validací formuláře. Zajišťuje bezpečný vstup uživatelských údajů a zobrazení chybových hlášení.

Stránka obsahuje:
\begin{itemize}
  \item Formulář s validací e-mailu a hesla
  \item Odkaz na registraci
  \item Zobrazení chybových zpráv
\end{itemize}

\begin{lstlisting}[style=csh, caption={Ukázka zpracování přihlášení uživatele.}]
Future<void> _onSubmit() async {
  if (!_formKey.currentState!.validate()) return;
  setState(() => _isSubmitting = true);

  try {
    final email = _emailController.text.trim();
    final password = _passwordController.text;
    await AuthService.instance.signIn(email: email, password: password);

    if (!mounted) return;
    setState(() => _isSubmitting = false);
    _navigateToHome();
  } on AuthException catch (e) {
    if (!mounted) return;
    setState(() => _isSubmitting = false);
    ScaffoldMessenger.of(context).showSnackBar(
      SnackBar(content: Text(e.message))
    );
  }
}
\end{lstlisting}

\textbf{Poznámka:} Validace a ošetření chyb zajišťují, že uživatel získá okamžitou zpětnou vazbu při nesprávném zadání údajů.

\begin{figure}[h]
  \centering
  \includegraphics[width=0.4\textwidth]{login_page.png}
  \caption{Ukázka rozhraní LoginPage.}
  \label{fig:signup_page}
\end{figure}


\section{SignUpPage}

\textit{SignUpPage} poskytuje uživateli rozhraní pro registraci nového účtu. Formulář zajišťuje validaci vstupních údajů a ošetření chyb.

Stránka obsahuje:
\begin{itemize}
  \item Formulář s validací:
    \begin{itemize}
      \item Jméno (povinné)
      \item E-mail (správný formát)
      \item Heslo (minimální délka)
      \item Potvrzení hesla (shoda s heslem)
    \end{itemize}
  \item Zobrazení chybových zpráv
\end{itemize}

\begin{lstlisting}[style=csh, caption={Ukázka zpracování registrace nového uživatele.}]
Future<void> _onSubmit() async {
  if (!_formKey.currentState!.validate()) return;

  try {
    await AuthService.instance.signUp(
      name: _nameController.text,
      email: _emailController.text,
      password: _passwordController.text,
    );
    Navigator.of(context).pop();
  } on AuthException catch (e) {
    ScaffoldMessenger.of(context).showSnackBar(
      SnackBar(content: Text(e.message))
    );
  }
}
\end{lstlisting}

\textbf{Poznámka:} Validace a ošetření chyb zajišťují bezpečný a přehledný proces registrace.

\begin{figure}[h]
  \centering
  \includegraphics[width=0.4\textwidth]{signup_page.png}
  \caption{Ukázka rozhraní SignUpPage.}
  \label{fig:signup_page}
\end{figure}


\newpage


\chapter*{Závěr}
\addcontentsline{toc}{chapter}{Závěr}
Cílem projektu bylo vytvořit mobilní~aplikaci, která bude obsahovat základní~nástroje pro kytaristy. Aplikace je postavená na~frameworku Flutter, a~je celý napsaný v~programovacím~jazyku Dart. Databáze je řešena přes Firestore~Database, a~autentizace přes Firebase~Authentication. Od začátku jsem se snažil aplikaci navrhnout tak, aby byla co nejvíce jednoduchá, a~srozumitelná. Výsledkem je aplikace, která nabízí plně funkční~metronom, ladičku, a~přehlednou databázi~akordů.


Odkaz na projekt: \url{https://github.com/JakubHalama/Fretfly-Maturitni_projekt}


%% Seznam použitých informačních zdrojů
\renewcommand\bibname{Seznam použitých informačních zdrojů}
\begin{thebibliography}{99}
\addcontentsline{toc}{chapter}{Seznam použitých informačních zdrojů}

\bibitem{flutterDocs}
\textit{Flutter — Documentation} [online]. Oficiální dokumentace frameworku Flutter. [cit. 2025-01-05]. Dostupné z: \url{https://docs.flutter.dev}

\bibitem{flutterTutorial}
\textit{Flutter Tutorial — Build apps for any screen} [online]. Oficiální výukové materiály Flutteru. [cit. 2025-01-05]. Dostupné z: \url{https://docs.flutter.dev/learn}

\bibitem{dartDocs}
\textit{Dart — Language Documentation} [online]. Oficiální dokumentace programovacího jazyka Dart. [cit. {2025-01-05}]. Dostupné z: \url{https://dart.dev/guides}

\bibitem{firebaseFlutter}
\textit{Firebase for Flutter — Documentation} [online]. Dokumentace k propojení Flutter aplikací se službami Firebase. [cit. 2025-01-05]. Dostupné z: \url{https://firebase.google.com/docs/flutter}

\bibitem{firebaseSetup}
\textit{Add Firebase to your Flutter app} [online]. Návod na integraci Firebase do Flutter projektu pomocí FlutterFire CLI. [cit. 2025-01-05]. Dostupné z: \url{https://firebase.google.com/docs/flutter/setup}

\bibitem{firebaseCodelab}
\textit{Get to know Firebase for Flutter — Codelab} [online]. Praktický kurz práce s autentizací a databází Firestore ve Flutter aplikaci. [cit. 2025-01-05]. Dostupné z: \url{https://firebase.google.com/codelabs/firebase-get-to-know-flutter}

\bibitem{flutterLayout}
\textit{Building layouts in Flutter} [online]. Návod na tvorbu uživatelského rozhraní pomocí widgetů. [cit. 2025-01-05]. Dostupné z: \url{https://docs.flutter.dev/ui/layout}

\bibitem{stateManagement}
\textit{State management — Flutter Documentation} [online]. Přehled přístupů ke správě stavu aplikace ve Flutteru. [cit. 2025-01-05]. Dostupné z: \url{https://docs.flutter.dev/data-and-backend/state-mgmt}

\bibitem{flutterTesting}
\textit{Testing Flutter apps} [online]. Oficiální dokumentace testování Flutter aplikací. [cit. 2025-01-05]. Dostupné z: \url{https://docs.flutter.dev/testing}

\bibitem{flutterFirebaseVideo}
\textit{Flutter and Firebase App Build — Full Tutorial} [online]. YouTube, 12.6.2022 [cit. 2025-01-05]. Dostupné z: \url{https://www.youtube.com/watch?v=1gDhl4leEzA}

\bibitem{flutterAuthVideo}
\textit{Flutter Firebase Authentication Tutorial} [online]. YouTube, 18.3.2021 [cit. 2025-01-05]. Dostupné z: \url{https://www.youtube.com/watch?v=7gqCzGgQG0U}

\bibitem{flutterFirestoreVideo}
\textit{Flutter Firestore CRUD Tutorial} [online]. YouTube, 5.9.2021 [cit. 2025-01-05]. Dostupné z: \url{https://www.youtube.com/watch?v=DqJ_KjFzL9I}

\end{thebibliography}

%% obrázky 
\newpage 
\listoffigures
\lstlistoflistings



\end{document}
